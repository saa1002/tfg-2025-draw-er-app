\capitulo{2}{Objetivos del proyecto}

En este apartado se detallan los objetivos a completar en el desarrollo del proyecto. Distinguiremos los marcados por los requisitos software y los objetivos de carácter técnico necesarios para completar el desarrollo del proyecto.

\section{Objetivos generales}
\begin{itemize}
\tightlist
    \item Desarrollar una aplicación web para modelar diagramas básicos entidad-relación.
    \item Proporcionar una interfaz sencilla y fácil de usar para el modelado de diagramas.
    \item Implementar funcionalidades para añadir y manipular entidades y relaciones.
    \item Permitir la validación de los diagramas modelados.
    \item Permitir exportar e importar los diagramas modelados como JSON.
    \item Exportar los diagramas validados a scripts SQL.
\end{itemize}

\section{Objetivos técnicos}
\begin{itemize}
\tightlist
    \item Utilizar React para el desarrollo de la aplicación web.
    \item Integrar la librería mxGraph para la creación y manipulación de diagramas.
    \item Implementar un sistema de menús contextuales para facilitar la manipulación de elementos del diagrama.
    \item Almacenar los datos de los diagramas en una estructura interna eficiente con las siguientes características:
    \begin{itemize}
        \item Es fácilmente procesable para la generación de las tablas SQL correspondientes.
        \item Es fácilmente extensible para expandir las interrelaciones que pueden modelarse.
    \end{itemize}
    \item Implementar la funcionalidad para validación de diagramas entidad-relación.
    \item Desarrollar la funcionalidad de exportación de diagramas a scripts SQL.
    \item Integrar pruebas unitarias y pruebas end-to-end.
    \item Configurar un flujo CI/CD que ejecute las pruebas y realice un despliegue automático de la aplicación en la plataforma Vercel.
    \item Utilizar git como sistema de control de versiones, usando como repositorio remoto la plataforma GitHub.
\end{itemize}