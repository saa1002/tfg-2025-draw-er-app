\apendice{Especificación de Requisitos}

\section{Introducción}
En este apartado se enumeran y desarrollan los objetivos y requisitos que debe tener la aplicación según han sido marcados al comienzo del proyecto.

\section{Objetivos generales}
Este proyecto tiene como objetivo los siguientes puntos:
\begin{itemize}
\tightlist
\item Modelar diagramas E-R.
\item Almacenar los diagramas en una estructura interna fácilmente extensible.
\item Validar esa estructura interna para conocer si un diagrama E-R es válido.
\item Exportar un diagrama válido a JSON.
\item Importar y recrear un diagrama en JSON válido.
\item Exportar los diagramas válidos a scripts SQL.
\end{itemize}
\newpage

\section{Catálogo de requisitos}
En este apartado se definirán los requisitos funcionales y no funcionales del proyecto.

\subsection{Requisitos funcionales}
\begin{itemize}
\tightlist
\item \textbf{RF1 - Modelar diagramas E-R:} La aplicación debe permitir a los usuarios crear y manipular diagramas E-R.
\item \textbf{RF2 - Almacenar diagramas:} La aplicación debe almacenar los diagramas en una estructura interna que sea fácilmente extensible.
\item \textbf{RF3 - Validar diagramas:} La aplicación debe validar la estructura interna del diagrama para asegurar su corrección al generar el SQL o exportar/importar JSON.
\item \textbf{RF4 - Exportar diagramas:} La aplicación debe permitir exportar los diagramas validados a un archivo JSON.
\item \textbf{RF5 - Importar diagramas:} La aplicación debe permitir importar diagramas desde un archivo JSON y recrearlos en el entorno de la aplicación.
\item \textbf{RF6 - Generar SQL:} La aplicación debe permitir generar scripts SQL a partir de los diagramas validados.
\end{itemize}

\newpage

\subsection{Requisitos no funcionales}
\begin{itemize}
\tightlist
\item \textbf{RNF1 - Usabilidad:} La aplicación debe ser fácil de usar y tener una interfaz intuitiva para mejorar la experiencia del usuario.
\item \textbf{RNF2 - Rendimiento:} La aplicación debe ser eficiente en términos de rendimiento.
\item \textbf{RNF3 - Compatibilidad:} La aplicación debe ser compatible con diferentes navegadores.
\item \textbf{RNF4 - Seguridad:} La aplicación debe garantizar la seguridad de los datos manejados y almacenados.
\item \textbf{RNF5 - Mantenibilidad:} La aplicación debe ser fácil de mantener y actualizar. Debe poder expandirse en el futuro con nuevos tipos de relaciones y funcionalidades.
\end{itemize}
\clearpage

\section{Especificación de requisitos}

\subsection{Casos de uso}

\begin{table}[p]
	\centering
	\begin{tabularx}{\linewidth}{ p{0.21\columnwidth} p{0.71\columnwidth} }
		\toprule
		\textbf{CU-1}    & \textbf{Modelar diagramas E-R}\\
		\toprule
		\textbf{Versión}              & 1.0    \\
		\textbf{Autor}                & Rubén Maté Iturriaga \\
		\textbf{Requisitos asociados} & RF-1 \\
		\textbf{Descripción}          & Permite al usuario crear y manipular diagramas E-R. \\
		\textbf{Precondición}         & La aplicación debe estar abierta y funcionando. \\
		\textbf{Acciones}             &
		\begin{enumerate}
			\def\labelenumi{\arabic{enumi}.}
			\tightlist
			\item El usuario arrastra y suelta elementos (entidades, relaciones) al canvas.
            \item El usuario añade atributos a las entidades y relaciones (en caso de que fueran N:M).
            \item El usuario edita los nombres y atributos de las entidades y relaciones.
            \item El usuario vincula entidades configurando las relaciones y asignándolas cardinalidades. 
		\end{enumerate}\\
		\textbf{Postcondición}        & El diagrama se almacena en la estructura interna de la aplicación. \\
		\textbf{Excepciones}          & - \\
		\textbf{Importancia}          & Alta \\
		\bottomrule
	\end{tabularx}
	\caption{CU-1 Modelar diagramas E-R.}
\end{table}

\begin{table}[p]
	\centering
	\begin{tabularx}{\linewidth}{ p{0.21\columnwidth} p{0.71\columnwidth} }
		\toprule
		\textbf{CU-2}    & \textbf{Almacenar diagramas}\\
		\toprule
		\textbf{Versión}              & 1.0    \\
	\textbf{Autor}                & Rubén Maté Iturriaga \\
		\textbf{Requisitos asociados} & RF-2 \\
		\textbf{Descripción}          & Permite almacenar los diagramas creados en una estructura interna. \\
		\textbf{Precondición}         & El usuario debe haber creado un diagrama E-R. \\
		\textbf{Acciones}             &
		\begin{enumerate}
			\def\labelenumi{\arabic{enumi}.}
			\tightlist
            \item La estructura interna se mantiene actualizada automáticamente ante los diversos cambios y acciones.
		\end{enumerate}\\
		\textbf{Postcondición}        & El diagrama se almacena correctamente y puede ser recuperado. \\
		\textbf{Excepciones}          & Problemas de almacenamiento (mensaje de error). \\
		\textbf{Importancia}          & Alta \\
		\bottomrule
	\end{tabularx}
	\caption{CU-2 Almacenar diagramas.}
\end{table}

\begin{table}[p]
	\centering
	\begin{tabularx}{\linewidth}{ p{0.21\columnwidth} p{0.71\columnwidth} }
		\toprule
		\textbf{CU-3}    & \textbf{Validar diagramas}\\
		\toprule
		\textbf{Versión}              & 1.0    \\
	\textbf{Autor}                & Rubén Maté Iturriaga \\
		\textbf{Requisitos asociados} & RF-3 \\
		\textbf{Descripción}          & Permite validar la estructura interna del diagrama para asegurar su corrección al generar el SQL o exportar/importar JSON. \\
		\textbf{Precondición}         & El usuario debe haber creado y modelado un diagrama. \\
		\textbf{Acciones}             &
		\begin{enumerate}
			\def\labelenumi{\arabic{enumi}.}
			\tightlist
			\item El usuario selecciona la opción de generar SQL o exportar/importar diagrama.
			\item La aplicación verifica la corrección del diagrama según las reglas E-R.
		\end{enumerate}\\
		\textbf{Postcondición}        & La aplicación muestra un mensaje indicando si el diagrama es válido o no antes de generar el SQL o exportar/importar JSON. \\
		\textbf{Excepciones}          & Mensajes de error con diagnóstico en caso de diagrama no válido. \\
		\textbf{Importancia}          & Alta \\
		\bottomrule
	\end{tabularx}
	\caption{CU-3 Validar diagramas.}
\end{table}

\begin{table}[p]
	\centering
	\begin{tabularx}{\linewidth}{ p{0.21\columnwidth} p{0.71\columnwidth} }
		\toprule
		\textbf{CU-4}    & \textbf{Exportar diagramas}\\
		\toprule
		\textbf{Versión}              & 1.0    \\
	\textbf{Autor}                & Rubén Maté Iturriaga \\
		\textbf{Requisitos asociados} & RF-4 \\
		\textbf{Descripción}          & Permite exportar los diagramas válidos a un archivo JSON. \\
		\textbf{Precondición}         & El diagrama debe ser válido. \\
		\textbf{Acciones}             &
		\begin{enumerate}
			\def\labelenumi{\arabic{enumi}.}
			\tightlist
			\item El usuario selecciona la opción de exportar diagrama.
			\item La aplicación genera un archivo JSON con la estructura interna del diagrama.
			\item El usuario descarga (la descarga se realiza automáticamente al pulsar el botón \texttt{Aceptar}) el archivo JSON.
		\end{enumerate}\\
		\textbf{Postcondición}        & El archivo JSON se guarda en el dispositivo del usuario. \\
		\textbf{Excepciones}          & Problemas de exportación (mensaje de error). \\
		\textbf{Importancia}          & Media \\
		\bottomrule
	\end{tabularx}
	\caption{CU-4 Exportar diagramas.}
\end{table}

\begin{table}[p]
	\centering
	\begin{tabularx}{\linewidth}{ p{0.21\columnwidth} p{0.71\columnwidth} }
		\toprule
		\textbf{CU-5}    & \textbf{Importar diagramas}\\
		\toprule
		\textbf{Versión}              & 1.0    \\
	\textbf{Autor}                & Rubén Maté Iturriaga \\
		\textbf{Requisitos asociados} & RF-5 \\
		\textbf{Descripción}          & Permite importar diagramas desde un archivo JSON y recrearlos en la aplicación. \\
		\textbf{Precondición}         & El usuario debe tener un archivo JSON válido. \\
		\textbf{Acciones}             &
		\begin{enumerate}
			\def\labelenumi{\arabic{enumi}.}
			\tightlist
			\item El usuario selecciona la opción de importar diagrama.
			\item El usuario carga el archivo JSON en la aplicación.
            \item La aplicación hace una validación del diagrama importado y si es correcta recrea el diagrama.
		\end{enumerate}\\
		\textbf{Postcondición}        & El diagrama se visualiza correctamente en la aplicación. \\
		\textbf{Excepciones}          & Problemas de importación (mensaje de error). \\
		\textbf{Importancia}          & Media \\
		\bottomrule
	\end{tabularx}
	\caption{CU-5 Importar diagramas.}
\end{table}

\begin{table}[p]
	\centering
	\begin{tabularx}{\linewidth}{ p{0.21\columnwidth} p{0.71\columnwidth} }
		\toprule
		\textbf{CU-6}    & \textbf{Generar SQL del diagrama modelado}\\
		\toprule
		\textbf{Versión}              & 1.0    \\
		\textbf{Autor}                & Rubén Maté Iturriaga \\
		\textbf{Requisitos asociados} & RF-6 \\
		\textbf{Descripción}          & Permite generar un script SQL a partir del diagrama E-R modelado y validado. \\
		\textbf{Precondición}         & El diagrama debe ser válido. \\
		\textbf{Acciones}             &
		\begin{enumerate}
			\def\labelenumi{\arabic{enumi}.}
			\tightlist
			\item El usuario selecciona la opción de generar SQL.
			\item La aplicación valida el diagrama E-R.
			\item La aplicación genera un script SQL con el paso a tablas del diagrama.
			\item El usuario descarga el archivo SQL (La descarga se realiza automáticamente al pulsar el botón \texttt{Aceptar}).
		\end{enumerate}\\
		\textbf{Postcondición}        & El archivo SQL se guarda en el dispositivo del usuario. \\
		\textbf{Excepciones}          & Problemas de generación de SQL (mensaje de error). \\
		\textbf{Importancia}          & Media \\
		\bottomrule
	\end{tabularx}
	\caption{CU-6 Generar SQL del diagrama modelado.}
\end{table}

