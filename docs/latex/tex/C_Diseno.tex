\apendice{Especificación de diseño}

\section{Introducción}

En este apéndice se detallan las decisiones de diseño tomadas durante el desarrollo. Se cubren los aspectos relacionados con el diseño de datos, el diseño procedimental y el diseño arquitectónico.

\section{Diseño de datos}
Esta aplicación no tiene como tal una estructura de datos definida puesto que no cuenta con un backend ni base de datos. Sin embargo sí que tiene la estructura JSON interna que contiene nuestro diagrama.

\begin{figure}
\lstset{breaklines=true, basicstyle=\footnotesize}
\begin{lstlisting}[frame=single]
{
    "entities": [
        {
            "idMx": "2",
            "name": "Entidad",
            "position": {
                "x": 254,
                "y": 130
            },
            "attributes": [
                {
                    "idMx": "3",
                    "key": false,
                    "name": "Atributo",
                    "position": {
                        "x": 374,
                        "y": 130
                    }
                },

            ]
        }
    ],
    "relations": [
    {
      "idMx": "8",
      "name": "Relacion",
      "position": {
        "x": 215,
        "y": 110
      },
      "side1": {
        "idMx": "11",
        "cardinality": "0:1",
        "cell": "11",
        "entity": {
          "idMx": "2"
        }
      },
      "side2": {
        "idMx": "12",
        "cardinality": "1:N",
        "cell": "12",
        "entity": {
          "idMx": "2"
        }
      },
      "canHoldAttributes": false,
      "attributes": []
    }
  ]
}
\end{lstlisting}
\caption{Representación interna de un ejemplo de diagrama E-R.}
\label{fig:e-r-structure}
\end{figure}

En esta estructura tenemos 3 entidades definidas que son: Entidades, Relaciones y Atributos.

En la figura \ref{fig:e-r-structure} podemos ver un ejemplo de un diagrama que constaría de una entidad llamada \texttt{Entidad} cuyos campos son:
\begin{itemize}
    \item idMx: Vinculación de id con la librería de renderización mxGraph.
    \item position: Coordenadas de esta entidad en el canvas.
    \item attributes: Colección de atributos que pertenecen a esta entidad de los cuáles se guardan a su vez:
    \begin{itemize}
        \item idMx.
        \item key: Si son clave o no.
        \item position.
    \end{itemize}
\end{itemize}
También tenemos una relación llamada \texttt{Relación} cuyos campos son:
\begin{itemize}
    \item idMx: Vinculación de id con la librería de renderización mxGraph.
    \item position: Coordenadas de esta entidad en el canvas.
    \item canHoldAttributes: Booleano que determina si esta relación puede guardar atributos, es decir si es N:M.
    \item attributes: Colección de atributos que pertenecen a esta relación (solo en caso de ser N:M) de los cuáles se guardan a su vez:
    \begin{itemize}
        \item idMx.
        \item key: Si son clave o no.
        \item position.
    \end{itemize}
    \item side1:
    \begin{itemize}
        \item idMx: id de la librería mxGraph que corresponde a la celda donde se guarda el label de la cardinalidad.
        \item cardinality: Cardinalidad de este lado de la relación.
        \item cell: Este campo es equivalente a idMx pero se conserva por compatibilidad.
        \item entity.idMx: id de la librería mxGraph que corresponde a la celda que contiene la entidad de esta lado de la relación.
    \end{itemize}
    \item side2: Con campos idénticos, corresponde al otro lado de la relación.
\end{itemize}

Su diagrama E-R es el siguiente: \ref{fig:structure-er}
\imagen{structure-er}{Diagrama E-R que representa la estructura interna sobra la que funciona la aplicación.}

Tanto Entidades como Relaciones pueden tener atributos asociados. En el caso de las entidades han de tener sí o sí un atributo, mientras que las relaciones pueden no tenerlo.
Las Relaciones pueden tener 1 o más lados, estando cada uno de estos lados relacionados con 1 Entidad. Una Entidad puede aparecer suelta en un diagrama E-R asi que no tiene porqué estar asociada a un Lado.

\section{Diseño procedimental}
En esta sección se describen los procesos más complejos de la aplicación utilizando diagramas de flujo para entenderlos más fácilmente.

En general ninguno de los procesos de la aplicación requiere grandes diagramas de flujo para comprenderlos. Sin embargo, sí que hay dos funcionalidades que ameritan el tener sus propios diagramas de flujo.

\subsection{Mover objetos}
Es una de las partes más complejas puesto que mover objetos ha de activar diversas tareas según el tipo de objeto que se esté moviendo.
Los objetos pueden ser:
\begin{itemize}
    \item Entidades.
    \item Relaciones.
    \begin{itemize}
        \item Reflexivas.
        \item No reflexivas.
    \end{itemize}
    \item Atributos.
\end{itemize}

\imagen{flow-move-object}{Diagrama de flujo que representa la funcionalidad de mover objetos en el canvas de la aplicación.}

\subsection{Generar el script SQL}
El script SQL es una de las funcionalidades más complejas de la aplicación. Requiere, aparte de la validación, un gran número de pasos y tablas intermedias donde se procesan los diferentes tipos de relaciones.
\imagencontamano{flow-generate-sql}{Diagrama de flujo que representa la funcionalidad de generar el script SQL en la aplicación.}{0.6}


\section{Diseño arquitectónico}
El diseño arquitectónico de la aplicación es bastante simple, puesto que se basa en una arquitectura frontend utilizando React y la librería mxGraph para el modelado de los diagramas E-R.

\subsection{Estructura de paquetes}
La estructura de paquetes de la aplicación se organiza de la siguiente manera:

\imagen{package-diagram}{Estructura de paquetes de la aplicación.}
\begin{itemize}
    \item \textbf{src}: Contiene todo el código fuente de la aplicación.
    \begin{itemize}
        \item \textbf{DiagramEditor}
        \begin{itemize}
            \item \textbf{DiagramEditor}: Componente principal de la aplicación.
            \item \textbf{initMxGraph}: Inicialización de la librería mxGraph.
            \item \textbf{toolbar}: Inicialización de la barra de herramientas.
            \end{itemize}
        \item \textbf{utils}:
        \begin{itemize}
            \item \textbf{validation}: Módulo que valida la estructura interna E-R.
            \item \textbf{sql}: Módulo que procesa y convierte la estructura interna E-R en un script SQL.
        \end{itemize}
    \end{itemize}
    \item App.js: Punto de entrada de la aplicación React.
\end{itemize}

\subsection{Componentes de la arquitectura}
Aunque la aplicación es puramente frontend, sí se pueden apreciar ciertos componentes externos a la aplicación:
\begin{itemize}
    \item \textbf{Archivos JSON}: Utilizados para la persistencia de los diagramas. Pueden ser importados.
    \item \textbf{LocalStorage}: Almacenamiento local del navegador donde se guarda el estado del diagrama que cuyo modelado esté en curso.
    \item \textbf{Navegador}: Medio a través del cual el usuario interactúa con la aplicación.
\end{itemize}

El siguiente diagrama ilustra estos componentes y su interacción con la aplicación:

\imagen{architecture}{Diagrama con las partes de la aplicación en su ejecución en el navegador.}
